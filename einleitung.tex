\section{Einleitung}
\subsection{Aufgabenstellung}

Im Rahmen des Moduls DBA02 war eine datenbankgestützte PHP--Anwendung für eine Website zu erstellen, welche die folgenden Kriterien erfüllt:

\begin{itemize}
\item \textbf{Frage stellen}: Besuchern der Website soll eine Frage gestellt werden, auf die sie mit einer oder mehreren vorgegebenen Möglichkeiten antworten können.
\item \textbf{Auswertung}: Nach der Beantwortung der Frage soll dem Besucher eine Auswertung der bisher gegebenen Antworten (Angaben in Prozent) gezeigt werden.
\item \textbf{Benutzerverwaltung}: Ein Administrator soll sich bei der Anwendung anmelden können. Hierzu soll ein Benutzername und Passwort abgefragt und geprüft werden.
\item \textbf{Neue Fragen eingeben}: Dem Seitenadministrator soll es über ein Formular möglich sein, neue Fragen mit den zugehörigen Antwortmöglichkeiten einzugeben. Normalen Besucher der Website ist diese Möglichkeit zu verwehren.  
\item \textbf{Datenbank}: Alle benötigten Daten werden in einer MySQL--Datenbank gespeichert.
\item \textbf{Echtzeitstatistiken}: Die Auswertung der gegebenen Antworten soll unmittelbar vor der Anzeige berechnet werden.
\item \textbf{XAMP}: Die Anwendung soll mit der Kombination von Apache--Webserver, MySQL--Datenbank und PHP als Programiersprache lauffähig sein. Das Betriebssystem kann frei gewählt werden. 
\end{itemize}


\subsection{Gemeinschaftsarbeit}

Die Aufgabe war arbeitsteilig in Teamarbeit zu lösen. Der Autor erstellte die für die Benutzerverwaltung und Frageneingabe notwengide Programmteile. Die Abfrage-- und Auswertungsseiten wurden von Yvonne Frezel gefertigt.

Das der Anwendung zu Grunde liegende Datenmodell wurde gemeinsam in einer Teambesprechung erarbeitet und festgelegt.

In diesem Assignment werden die vom Autor erstellten Programmteile dokumentiert.

\subsection{Abgrenzung}

Da eine Datenbankgestützte Web--Anwendung in der Regen einem großen Personenkreis\footnote{allen Internet-- oder Intranetnutzern} zur Verfügung steht sind hier unbedingt Sicherheitsaspekte zu beachten. Da eine ausführliche Betrachtung dieser Maßnahmen den Rahmen dieses Dokuments sprechengen würde, werden nur entsprechende Hinweise auf weiterführende Informationen gegeben.