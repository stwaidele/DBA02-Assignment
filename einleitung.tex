\section{Einleitung}
\subsection{Aufgabenstellung}

Im Rahmen des Moduls DBA02 war eine datenbankgestützte PHP--Anwendung für eine Website zu erstellen, welche die folgenden Kriterien erfüllt:

\begin{itemize}
\item \textbf{Frage stellen}: Besuchern der Website soll eine Frage gestellt werden, auf die sie mit einer oder mehreren vorgegebenen Möglichkeiten antworten können.
\item \textbf{Auswertung}: Nach der Beantwortung der Frage soll dem Besucher eine Auswertung der bisher gegebenen Antworten (Angaben in Prozent) gezeigt werden.
\item \textbf{Benutzerverwaltung}: Ein Administrator soll sich bei der Anwendung anmelden können. Hierzu soll ein Benutzername und Passwort abgefragt und geprüft werden.
\item \textbf{Neue Fragen eingeben}: Dem Seitenadministrator soll es über ein Formular möglich sein, neue Fragen mit den zugehörigen Antwortmöglichkeiten einzugeben. Normalen Besucher der Website ist diese Möglichkeit zu verwehren.  
\item \textbf{Datenbank}: Alle benötigten Daten werden in einer MySQL--Datenbank gespeichert.
\item \textbf{Echtzeitstatistiken}: Die Auswertung der gegebenen Antworten soll unmittelbar vor der Anzeige berechnet werden.
\item \textbf{XAMP}: Die Anwendung soll mit der Kombination von Apache--Webserver, MySQL--Datenbank und PHP als Programiersprache lauffähig sein. Das Betriebssystem kann frei gewählt werden. 
\end{itemize}

Desweiteren sollte die Anwendung objektorientiert programmiert werden, Enturfsmuster verwenden und die HTML--Ausgabe per CSS formtiert werden.

\subsection{Gemeinschaftsarbeit}

Die Aufgabe war arbeitsteilig in Teamarbeit zu lösen. Das der Anwendung zu Grunde liegende Datenmodell wurde gemeinsam in einer Teambesprechung erarbeitet und festgelegt. Anschließend wurde ein Mockup\footnote{engl. für Attrappe. (to mock: nachahmen)} der HTML--Seiten und die benötigten SQL--Abfragen, gefolgt von einem prozedural programmierten Prototypen erstellt. Hierbei erstellte der Autor die für die Benutzerverwaltung und Frageneingabe notwendige Programmteile. Die Abfrage-- und Auswertungsseiten wurden von Yvonne Frezel gefertigt.

Da die im Seminar DBA02 begonnene Umsetzung des Programmcodes in Klassen unterschiedliche Richtungen verfolgte, wurde die enge Teamarbeit anschließend nicht mehr weitergeführt. Aufgrund der gemeinsamen Datenbasis sind die vom Author und von Frenzel erstellten PHP--Dateien miteinander kombinierbar, auch wenn sie intern andere Klassen und Zugriffsmethoden nutzen.

Dieses Assignment geht hauptsächlich auf die vom Autor konzipierten Programmteile „Benutzerverwaltung“ und „Neue Fragen hinzufügen“ ein.

\subsection{Aufbau der Arbeit}

Zunächst wurden die für die Anwendungen relevanten Konzepte, Techniken, und Frameworks beschrieben. Anschließend erfolgte die Beschreibung der Implementierungsdetails und der vom Autor gewählten Lösungsmöglichkeiten  

\subsection{Abgrenzung}

Da eine datenbankgestützte Web--Anwendung in der Regel einem großen Personenkreis\footnote{allen Internet-- oder zumindest Intranetnutzern} zur Verfügung steht sind hier unbedingt Sicherheitsaspekte zu beachten. Da eine ausführliche Betrachtung dieser Maßnahmen den Rahmen dieses Dokuments sprechengen würde, werden nur entsprechende Hinweise auf weiterführende Informationen gegeben. Auch Performance--Überlegungen gehen nur in sehr beschränktem Maß in die Implementation ein.
