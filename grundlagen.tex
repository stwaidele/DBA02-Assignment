\section{Datenbank--Schema}

\subsection{Tabelle: user}
\begin{center}
\tablehead{ \textbf{user} & & 
\\ }
\bottomcaption[Beschreibung]{Beschreibung. Quelle: Berger, Vorlesung, 2012, München }
\begin{supertabular}{c|c|c}
\hline
email & varchar(255) &  \\
pw & char(32) &  \\
\end{supertabular}
\end{center}


\begin{figure}[h]
\begin{minted}[bgcolor=bg]{sql}
CREATE TABLE user (
  email VARCHAR(255) NOT NULL,
  pw CHAR(32) NOT NULL,
  create_time TIMESTAMP DEFAULT CURRENT_TIMESTAMP,
  PRIMARY KEY (`email`));
\end{minted}
\caption{SQL: CREATE TABLE user}
\label{sql:tbluser}
\end{figure}

Als Benutzername wird hierbei die E--Mail Adresse des Administrators genutzt. 

Das Passwort sollte nicht im Klartext in der Datenbank gespeichert werden. Ein \code{salted hash}\footnote{Also der Hashwert des Passworts, welches zuvor mit Applikationsspezifischen Zusatzdaten ergänzt wurde} schützt hier das Passwort vor dem Ausspähen durch den Administrator selbst\footnote{Seit Vodafone 2013 eine dokumentierte Gefahr} oder durch Angreifer.\\
Die hier reservierten 32 Byte sind für den in der MySQL--Dokumentation\footnote{\cite{mysql-pcrypt}} emphohlenen MD5-Hash ausreichend. Da die entsprechende PHP--Dokumentation\footnote{\cite{php-pcrypt}} hier allerdings eine genau entgegengesetzte Empfehlung gibt, ist dieser Sicherheitsaspekt für ein Produktivsystem nochmals genauer zu prüfen.

\subsection{Tabelle: frage}
\begin{figure}[h]
\begin{minted}[bgcolor=bg]{sql}
CREATE TABLE frage (
  fid INT NOT NULL AUTO_INCREMENT,
  txt VARCHAR(1024) NOT NULL,
  PRIMARY KEY (`fid`));
\end{minted}
\caption{SQL: CREATE TABLE frage}
\label{sql:tblfrage}
\end{figure}

\subsection{Tabelle: antwort}
\begin{figure}[h]
\begin{minted}[bgcolor=bg]{sql}
CREATE TABLE antwort (
  aid INT NOT NULL AUTO_INCREMENT,
  nr  INT NULL,
  txt VARCHAR(1024) NOT NULL,
  fid INT NOT NULL,
  PRIMARY KEY (`aid`),
  FOREIGN KEY (`fid`) REFERENCES frage(`fid`));
\end{minted}
\caption{SQL: CREATE TABLE antwort}
\label{sql:tblantwort}
\end{figure}

\subsection{Tabelle: geantwortet}
\begin{figure}[h]
\begin{minted}[bgcolor=bg]{sql}
CREATE TABLE geantwortet (
  gid INT NOT NULL AUTO_INCREMENT,
  aid INT NOT NULL,
  zs TIMESTAMP DEFAULT CURRENT_TIMESTAMP,
  PRIMARY KEY (`gid`),
  FOREIGN KEY (`aid`) REFERENCES antwort(`aid`));
\end{minted}
\caption{SQL: CREATE TABLE geantwortet}
\label{sql:tblgeantwortet}
\end{figure}

\subsection{Query: Frage -- Text}
Wird für die Abfrage und Auswertung benötigt.

\begin{figure}[h]
\begin{minted}[bgcolor=bg]{sql}
select txt from frage where fid = <n>;
\end{minted}
\caption{SQL: Text von Frage <n>}
\label{sql:qfragetxt}
\end{figure}

\subsection{Query: Antwortmöglichkeiten}
Wird für die Abfrage benötigt.

\begin{figure}[h]
\begin{minted}[bgcolor=bg]{sql}
select antwort.txt
	from antwort
	where fid = <n> 
\end{minted}
\caption{SQL: Mögliche Antworten für Frage <n>}
\label{sql:qantwnum}
\end{figure}


\subsection{Query: Anzahl der gegebenen Antworten}
Wird für die Auswertung benötigt: 100\%

\begin{figure}[h]
\begin{minted}[bgcolor=bg]{sql}
select count(geantwortet.aid) 
	from antwort, geantwortet 
	where geantwortet.aid = antwort.aid and antwort.fid = <n>;
\end{minted}
\caption{SQL: Antwortzahl 100\% von Frage <n>}
\label{sql:qantw100}
\end{figure}

\subsection{Query: Gegebene Antworten und Häufigkeit}
Wird für die Auswertung benötigt.

\begin{figure}[h]
\begin{minted}[bgcolor=bg]{sql}
select antwort.txt, count(geantwortet.aid) 
	from antwort, geantwortet 
	where geantwortet.aid = antwort.aid and antwort.fid = <n> 
	group by geantwortet.aid;
\end{minted}
\caption{SQL: Gegebene Antworten mit Häufigkeit für Frage <n>}
\label{sql:qantwnum}
\end{figure}
